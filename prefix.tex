% 中文
\usepackage{fontspec,xltxtra,xunicode}  
\usepackage{xeCJK} % For windows OS CJK
\defaultfontfeatures{Mapping=tex-text}  
%\setromanfont{SimSun} %中文
\XeTeXlinebreaklocale “zh”  
\XeTeXlinebreakskip = 0pt plus 1pt minus 0.1pt 
\setmainfont{Times New Roman} % For windows OS CJK
\setCJKmainfont[AutoFakeBold=1,AutoFakeSlant=.3]{標楷體} % For windows OS CJK

\newCJKfontfamily\Kai{標楷體}       %定義指令\Kai則切換成標楷體
\newCJKfontfamily\Hei{微軟正黑體}   %定義指令\Hei則切換成正黑體
\newCJKfontfamily\NewMing{新細明體} %定義指令\NewMing則切換成新細明體
\newCJKfontfamily\WeiBei{魏碑-繁} %定義指令\NewMing則切換成新細明體


\usepackage{url}
\usepackage[bookmarks]{hyperref}
\usepackage{tabularx}
\usepackage{comment}
\usepackage{amsthm}
\usepackage{thmtools}
\usepackage{framed} %有框的文字
\usepackage{caption}
\usepackage{subcaption}
\usepackage{longtable}
\usepackage{makeidx} 
\usepackage{amsmath} 
\usepackage{centernot} %\not implies
\usepackage{csvsimple}
\usepackage{tikz} % include icon
\usepackage{blindtext}
\usepackage{tcolorbox}
\usepackage{minitoc} % toc in each chapter
\usepackage[cc]{titlepic}% title image
\tcbuselibrary{breakable}
\usepackage[inline, shortlabels]{enumitem}
\usepackage{wrapfig}



% itemize and enumerate
\usepackage{enumitem}
\setlist[enumerate]{itemsep=-2pt,topsep=3pt}
\setlist[itemize]{itemsep=-2pt,topsep=3pt}


% 圖
\usepackage{graphicx}
%\renewcommand\thesubsection{\Alph{subsection}} 
\renewcommand{\figurename}{圖}
\renewcommand{\tablename}{表}
\renewcommand\contentsname{目錄}
\renewcommand{\bibname}{References}

% path
\newcommand{\codePath}{./SampleCode/src/}
\newcommand{\codeURL}{https://github.com/nlhsueh/sqa/tree/master/src}

% for source code
\usepackage{listings}
	\newfontfamily\listingsfont[Scale=.8]{Menlo}
\lstset{breaklines=true,% 過長的程式行可斷行
tabsize=4,
extendedchars=false,% 中文處理不需要 extendedchars
texcl=true,% 中文註解需要有 TeX 處理過的 comment line, 所以設成 true
comment=[l]\%\%,% 以雙「百分號」做為程式中文註解的起頭標記,配合 MATLAB
basicstyle=\small,
%basicstyle=\listingsfont,
commentstyle=\upshape,% 預設是斜體字,會影響註解裏的英文,改用正體
%language=Octave % 會將一些 octave 指令以粗體顯示
  language=JAVA, 
  breaklines=true, 
  numbers=none, 
  numberstyle=\tiny,
  captionpos=t,
  stepnumber=1, 
  numbersep=10pt,
  commentstyle=\textit,
  stringstyle=\ttfamily,
  xleftmargin={2em},
  xrightmargin={1em},
  %lineskip=-1pt,
}
\renewcommand{\lstlistingname}{程式}
               
\renewenvironment{quote}
               {\list{}{\rightmargin=0.3in\leftmargin=0.3in}%
                \item \relax \Large  %\textquotedblleft  
                \normalsize} %\ignorespaces
               {\unskip\unskip  \endlist} %\textquotedblright
               

\declaretheorem[style=remark, name={}]{Example}               
         

%include PDF file
\usepackage{pdfpages}
\usepackage{float}      

%headline
%\newcommand{\headline}[1]{\noindent\textbf{#1}}

\hypersetup{
    bookmarks=true,         % show bookmarks bar?
    unicode=true,          % non-Latin characters in Acrobat’s bookmarks
    pdftoolbar=true,        % show Acrobat’s toolbar?
    pdfmenubar=true,        % show Acrobat’s menu?
    pdffitwindow=false,     % window fit to page when opened
    pdfstartview={FitH},    % fits the width of the page to the window
    pdftitle={SQA},    % title
    pdfauthor={薛念林},     % author
    pdfsubject={Software Quality},   % subject of the document
    pdfcreator={Creator},   % creator of the document
    pdfproducer={薛念林}, % producer of the document
    pdfkeywords={software, quality, testing}, % list of keywords
    pdfnewwindow=true,      % links in new PDF window
    colorlinks=true,       % false: boxed links; true: colored links
    linkcolor=black, %purple,          % color of internal links (change box color with linkbordercolor)
    citecolor=gray,        % color of links to bibliography
    filecolor=magenta,      % color of file links
    urlcolor=blue           % color of external links
}

\usepackage{graphicx}
\graphicspath{{./image/}}

% change color of section title

\usepackage{titlesec}
%\newcommand{\sectionbreak}{\clearpage}
%\newcommand{\subsectionbreak}{\clearpage}

\usepackage{xcolor}
\definecolor{MSBlue}{rgb}{.204,.353,.541}
\definecolor{MSLightBlue}{rgb}{.31,.506,.741}
\definecolor{airforceblue}{rgb}{0.36, 0.54, 0.66}
\definecolor{antiquewhite}{rgb}{0.98, 0.92, 0.84}
\definecolor{arsenic}{rgb}{0.23, 0.27, 0.29}
\titleformat*{\section}{\Large\bfseries\sffamily\color{MSBlue}}
\titleformat*{\subsection}{\large\bfseries\sffamily\color{MSLightBlue}}
%\titleformat*{\subsubsection}{\itshape\subsubsectionfont}

% ICON
\newcommand{\questionIcon}{\includegraphics[scale=1]{./icon/question-speech-bubble.png} \enspace} 

\newcommand{\BearIcon}{\enspace \includegraphics[scale=0.6]{./icon/bear.png} \enspace}
%\newcommand{\QuoteIcon}{\includegraphics[scale=0.6]{./icon/left179.png} \enspace}
\newcommand{\JokeIcon}{\includegraphics[scale=0.6]{./icon/mask17.png} \enspace}
%\newcommand{\Icon}{\includegraphics[scale=0.6]{./icon/right198.png} \enspace}
%\newcommand{\PrincipleIcon}{\includegraphics[scale=0.6]{./icon/fire.png} \enspace}
\newcommand{\ExIcon}{\includegraphics[scale=0.6]{./icon/exercise3.png} \enspace}
\newcommand{\SmallExIcon}{\includegraphics[scale=0.3]{./icon/exercise3.png} \enspace}
\newcommand{\importantIcon}{\includegraphics[scale=1]{./icon/move-to-next.png} \enspace} 


\newcommand{\ReportIcon}{\includegraphics[scale=0.5]{./icon/pen.png} \enspace}
\newcommand{\CodingIcon}{\includegraphics[scale=0.5]{./icon/computer.png} \enspace}
\newcommand{\ThinkIcon}{\includegraphics[scale=0.5]{./icon/think.png} \enspace}
\newcommand{\PenIcon}{\includegraphics[scale=0.5]{./icon/pen.png} \enspace}
\newcommand{\DiscussIcon}{\includegraphics[scale=0.5]{./icon/discuss.png} \enspace}
\newcommand{\LabIcon}{\includegraphics[scale=0.5]{./icon/lab.png} \enspace}
\newcommand{\FBIcon}{\includegraphics[scale=0.4]{./icon/facebook-logo.png} \enspace}
\newcommand{\GoogleGroupIcon}{\includegraphics[scale=0.4]{./icon/discuss.png}\enspace}
\newcommand{\ResearchIcon}{\includegraphics[scale=0.1]{./icon/lamp.png} \enspace}
\newcommand{\HandPenIcon}{\includegraphics[scale=0.05]{./icon/hand_pen.png} \enspace}
\newcommand{\VideoIcon}{\includegraphics[scale=0.08]{./icon/video.png} \enspace}

% Quote and Def
\newcommand{\storyend}{\rule{\linewidth}{0.5pt} }
\newcommand{\myquote}[3]{\begin{quote}\textbf{#1} #2  \textit{#3}\end{quote}}
\newcommand{\question}[1]{\begin{quote} \questionIcon {\WeiBei #1} \end{quote}}
\newcommand{\mquote}[1]{\begin{quote} \QuoteIcon #1 \end{quote}}
\newcommand{\myvideo}[2]{\begin{quote} \VideoIcon \href{#1}{#2} \end{quote}}

\newenvironment{story}{\verse \rule{\linewidth}{0.5pt} \\ 
\color{arsenic} }

\newcommand{\mydef}[2]{
\begin{center}\begin{tcolorbox}[title=\DefinitionIcon #1, width=0.85\columnwidth, title filled=false,colbacktitle=yellow, coltitle=black]
#2
\end{tcolorbox}\end{center}
}

% 練習題,參考本文
\newcommand{\RefSec}[1]{
%\footnotesize{(sec\ref{#1}/p\pageref{#1})}
\footnotesize{\hyperref[#1]{(參考本文)}}
}

\newcommand{\BBFB}{\href{https://www.facebook.com/groups/1678843235698942/}{\FBIcon {\footnotesize 加入討論}}}

\newcommand{\exerciseMargin}
{\newgeometry{top=0.75in, bottom=6in, left=.5in, right=.5in}}

\newcommand{\originalMargin}
{\newgeometry{top=0.75in, bottom=.5in, left=.5in, right=.5in}}

%\newcommand{\discuss}[1]{\href{https://groups.google.com/forum/#!forum/bigbear_nlhsueh}{\GoogleGroupIcon {\footnotesize 加入討論}}}


% Def of some variable
\newif\ifWide 					\Widefalse
\newif\ifAllChapter			\AllChapterfalse
\newif\ifAns 					\Ansfalse
\newif\ifShowContent 	\ShowContenttrue % show the article, not only lab
\newif\ifSlide					\Slidefalse
\newif\ifBook					\Booktrue
\newif\ifTBook				\TBookfalse % two column book
\newif\ifEx						\Exfalse % Exercise book
\newif \ifnotEx				\notExtrue





% =====================
% Old in SQA.tex
%\usepackage{amsthm}
%\usepackage{thmtools}
%\newtheorem{defi}{Definition}
%
%\usepackage{ragged2e}
%\usepackage{wrapfig,lipsum,booktabs}
%\usepackage{subcaption}
%\usepackage{wraptable}
% ==============================


